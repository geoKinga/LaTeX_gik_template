\newpage % Rozdziały zaczynamy od nowej strony.
\section{Rozdział 3}
W tym rozdziale też mogłoby być kilka podrozdziałów.


\subsection{Wstawianie kodu}
Ten rozdział będzie o wstawianiu kodu. Najcześciej fragmenty algorytmów w pracach dyplomowych przedstawiamy za pomocą \textbf{pseudokodu} lub \textbf{schematu blokowego}. Jednak czasami pojawia się koniecznośc wstawienia fragmentu algorytmu zimplementowanego w danym języku (np. Python) \citep{Teunissen.Montenbruck2017}. W tym celu w \LaTeX{} można skorzystać z biblioteki \emph{lstlisting} oraz zdefiniowac odpowedni styl kodu w preambule dokumentu (np. kolorowanie słów kluczowych). Więcej informacji na temat modyfikacji ustawiel otoczenia można znaleźć na portalu wikibooks \LaTeX{}\footnote{\url{https://en.wikibooks.org/wiki/LaTeX/Source_Code_Listings}} lub na oficjalnej stronie biblioteki.



Kod w liniejce można wstawić tak \lstinline|if x == 0:|

\begin{lstlisting}[language=Python,	caption={\emph{Przykładowy kod w Pythonie} }]
import numpy as np
import scipy.integrate as integrate
import matplotlib.pyplot as plt

# Our integral approximation function
def integral_approximation(f, a, b):
	return (b-a)*np.mean(f)

# Integrate f(x) = x^2
def f1(x):
	return x**2

# Define bounds of integral
a = 0
b = 1

# Generate function values
x_range = np.arange(a,b+0.0001,.0001)
fx = f1(x_range)

# Approximate integral
approx = integral_approximation(fx,a,b)
\end{lstlisting}



