\subsection{Formatowanie tekstu} 

\begin{itemize}
\item Automatyczne łamanie liniii stron, akapity oddzielane są od siebie pustymi liniami. Ilość użytych spacji i/lub pustych linii nie ma znaczenia.
\item "myślniki"
\begin{itemize}
    \item - dywiz (hyphen): \verb|-|
    \item -- łącznik (short dash): \verb|--|
    \item --- myślnik (long dash): \verb|---|
\end{itemize}
\item Rodzaj i grubość czcionki:
\begin{itemize}
    \item \verb|\textrm{tekst}| - \textrm{tekst} – pismo proste,
    \item \verb|\textsl{tekst}| - \textsl{tekst} – pismo proste pochylone
    \item \verb|\textit{tekst}| - \textit{tekst} – kursywa (italic),
    \item \verb|\emph{tekst}|   - \emph{tekst}    – wyróżnienie (zalecane),
    \item \verb|\textbf{tekst}| - \textbf{tekst} – pismo pogrubione (bold),
    \item \verb|\texttt{tekst}| - \texttt{tekst} – pismo imitujące pismo maszynowe
    \item \verb|\textsf{tekst}| - \textsf{tekst} – pismo bezszeryfowe
\end{itemize}
\item Wielkość czcionki definujemy w opcjach klasy dukumentu, czyli:
\begin{verbatim}
\documentclass[10pt,a4paper]{article}
\end{verbatim}
\item Nastepnie wielkość czcionki w stosunku do \emph{normalsize} zieniamy poprzez:
\begin{verbatim}
\tiny           % 5pt
\scriptsize     % 7pt
\footnotesize   % 8pt
\small          % 9pt
\normalsize     % 10pt
\large          % 12pt
\Large          % 14pt
\LARGE          % 17pt
\huge           % 20pt
\Huge           % 25pt
\end{verbatim}
\end{itemize}
