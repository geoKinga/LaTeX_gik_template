\newpage
\subsection{Rysunki}

\begin{figure}[!ht]
	\centering 
	\includegraphics[width=0.5\textwidth]{od_polinistow.png}
	\caption{Pozdrowienia od polonistów}\label{fig:polonistyka}
\end{figure}

\begin{figure}[!ht]
	\centering
	\begin{subfigure}[b]{0.45\textwidth}
		\centering
		\includegraphics[height=0.3\textheight,keepaspectratio]{poziom_istotnosci.jpg}
		\caption{Obrazek YX}\label{fig:obrazekA}
	\end{subfigure}
	~ %add desired spacing between images, e. g. ~, \quad, \qquad, \hfill etc. 
	%(or a blank line to force the subfigure onto a new line)
	\begin{subfigure}[b]{0.45\textwidth}
		\centering
		\includegraphics[height=0.3\textheight,keepaspectratio]{machine_learning.jpg}
		\caption{Obrazek XY}\label{fig:obrazekB}
	\end{subfigure}
	\caption{Dwa rysunki obok siebie}\label{fig:Obrazki}
\end{figure}

Odniesienia do rysunków np. Jak pokazano na rysunku \ref{fig:polonistyka} oraz \ref{fig:obrazekA}.
Natomiast rynunek \ref{fig:Obrazki} opisuje to samo zjawisko.



